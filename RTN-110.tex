\documentclass[OPS,lsstdraft,authoryear,toc]{lsstdoc}
\input{meta}

% Package imports go here.

% Local commands go here.

%If you want glossaries
%\input{aglossary.tex}
%\makeglossaries

\title{ToO Advisory Board Charge}

% This can write metadata into the PDF.
% Update keywords and author information as necessary.
\hypersetup{
    pdftitle={ToO Advisory Board Charge},
    pdfauthor={Robert D.Blum},
    pdfkeywords={}
}

% Optional subtitle
% \setDocSubtitle{A subtitle}

\input{authors}

\setDocRef{RTN-110}
\setDocUpstreamLocation{\url{https://github.com/lsst/rtn-110}}

\date{\vcsDate}

% Optional: name of the document's curator
% \setDocCurator{The Curator of this Document}

\setDocAbstract{%
The Rubin ToO Advisory Board is charged with giving advice on various aspects of the Target of Opportunity (ToO) process during Rubin Operations.
}

% Change history defined here.
% Order: oldest first.
% Fields: VERSION, DATE, DESCRIPTION, OWNER NAME.
% See LPM-51 for version number policy.
\setDocChangeRecord{%
  \addtohist{1}{2025-11-26}{Released.}{Blum}
}


\begin{document}

% Create the title page.
\maketitle
% Frequently for a technote we do not want a title page  uncomment this to remove the title page and changelog.
% use \mkshorttitle to remove the extra pages

The Rubin Target of Opportunity (ToO) Advisory Board is charged with giving advice on various aspects of the ToO process during Rubin Operations.

The Board will work directly with the Rubin ToO Observer (RTO), a 24/7/365 role responsible for managing triggers of ToOs at the observatory.
The RTO will be responsible for making trigger decisions in real time.

\section{Responsibilities}

While the triggering of ToO observations with Rubin Observatory will largely be by automated algorithms, the Board will give input to the process on three timescales.

\subsection{Short timescales (<24 hours)}

When a set of ToO observations is automatically triggered, all members of the Board would be immediately informed.
They would be invited to review the automatically planned set of observations, and advise the RTO on modifications, or even cancellations of the Rubin follow-up, e.g., as the properties of the trigger event become clearer.
They could also recommend that an event that falls below the threshold for follow-up observations is sufficiently promising to carry out Rubin observations.

\subsection{Medium timescales (days-weeks)}

Following any set of ToO-triggered Rubin observations, the Board will carry out a review to find out whether the observations were successful, and give feedback to the RTO and the observatory about the process.
They will also review any events from LIGO-Virgo-Kagra (LVK), IceCube or otherwise that didn’t trigger Rubin follow-up, and ask whether perhaps the trigger criteria should be expanded.
In both cases, the triggering algorithms and follow-up strategy could be modified.

\subsection{Long timescales (months-years)}

In coordination with the Rubin Survey Cadence Optimization Committee (SCOC), the Board would recommend changes to the ToO trigger criteria and follow-up strategy, as well as consider expansions of the scope of the program (such as including other science cases or triggers from other facilities) as scientific opportunities change and experience with the ToO process develops.

\section{Membership}

The Board membership process is run by the Rubin Science Advisory Committee with final approval of invitation to the Board by the Rubin Observatory Director.

The committee will be made up of experts in the following fields.

\begin{itemize}
\item Time domain astronomy and solar system science, including both scientific and technical aspects of each field.
\item Gravitational wave events and their electromagnetic counterparts.
\item Neutrino events and their electromagnetic counterparts.
\item Potentially Hazardous Asteroids (PHAs).
\end{itemize}

The Board should include representatives from the Chilean, UK, and French communities as significant operations partners.
Nominations from the broader international community of in-kind contribution teams are very welcome as well. 

Board members will serve for a two-year term, renewable once.
Half of the first cohort of members will end the term after the second year, with the other half extending their term for a second two-year term, to ensure continuity on the Board thereafter.

The Board names its own Chair from among its members.



\appendix

\section{Acknowledgements}

This material is based upon work supported in part by the National Science Foundation through Cooperative Agreements AST-1258333 and AST-2241526 and Cooperative Support Agreements AST-1202910 and AST-2211468 managed by the Association of Universities for Research in Astronomy (AURA), and the Department of Energy under Contract No.\ DE-AC02-76SF00515 with the SLAC National Accelerator Laboratory managed by Stanford University.
Additional Rubin Observatory funding comes from private donations, grants to universities, and in-kind support from LSST-DA Institutional Members.

% Include all the relevant bib files.
% https://lsst-texmf.lsst.io/lsstdoc.html#bibliographies
\section{References} \label{sec:bib}
\renewcommand{\refname}{} % Suppress default Bibliography section
\bibliography{local,lsst,lsst-dm,refs_ads,refs,books}

% Make sure lsst-texmf/bin/generateAcronyms.py is in your path
\section{Acronyms} \label{sec:acronyms}
\input{acronyms.tex}
% If you want glossary uncomment below -- comment out the two lines above
%\printglossaries





\end{document}
